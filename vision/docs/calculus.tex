\documentclass[a4paper]{article}
\usepackage[utf8]{inputenc}
\usepackage[french]{babel}
\usepackage[T1]{fontenc}
\usepackage{amsmath}
\usepackage{amssymb}
\usepackage{a4wide}
\usepackage{bm}

\begin{document}

\section{Calcul de la matrice de covariance des estimées des poses relatives des marqueurs AruCo}

\noindent Paramétrisation locale de la pose relative : on pose $\bm{\delta\tau} = \begin{bmatrix} \bm{\delta\theta}, \bm{\delta{p}} \end{bmatrix}^{\top}$ :
\begin{align}
\bm{\delta\tau} \boxplus \begin{bmatrix} \bm{R} & \bm{t} \\ \bm{0}_{1\times{3}} & 1 \end{bmatrix} =
\begin{bmatrix} \bm{\delta\theta} \boxplus \bm{R} & \bm{t} + \bm{\delta{p}} \\ \bm{0}_{1\times{3}} & 1 \end{bmatrix}
\end{align}
L'incrément en rotation est :
\begin{align}
\bm{\delta\theta} \boxplus \bm{R} = \exp\left(\begin{bmatrix}\bm{\delta\theta}\end{bmatrix}_{\times}\right) \cdot \bm{R}
\approx \left( \bm{I}_3 + \begin{bmatrix} \bm{\delta\theta} \end{bmatrix}_{\times} \right) \cdot \bm{R}
\end{align}
Modèle de projection du coin $c$ en coordonnées homogènes :
\begin{align}
{}^I\tilde{\bm{p}}_c(\bm{\delta\tau}) &= \bm{K} \cdot \left( \bm{\delta\tau} \boxplus T_{CM} \right) 
\cdot {}^M\tilde{\bm{p}}_c\\
&= \bm{K} \cdot \begin{bmatrix} \bm{\delta\theta} \boxplus \bm{R} & 
\bm{\delta{p}} + \bm{t} \\ \bm{0}_{1\times{3}} & 1 \end{bmatrix} \cdot \begin{bmatrix} 
{}^{\mathcal{I}}\bm{p}_c \\ 1 \end{bmatrix}\\
&= \bm{K} \cdot \begin{bmatrix} \left( \bm{\delta\theta} \boxplus \bm{R} \right) \cdot
{}^{\mathcal{I}}\bm{p}_c + (\bm{\delta{p}} + \bm{t}) \\ 1 \end{bmatrix}
\end{align}
Calcul de la matrice Jacobienne :
\begin{subequations}
\begin{align}
\bm{J}^{{}^I\tilde{\bm{p}}_c}_{T_{CM}} &= \left.\dfrac{\partial{}^I\bm{p}_c(\bm{\delta\tau})}{\partial\bm{\delta\tau}}
\right|_{\bm{\delta\tau}=0} \\
&= \begin{bmatrix} \left.\dfrac{\partial{}^{\mathcal{I}}\bm{p}_c}{\partial\bm{\delta\theta}} \right|_{\bm{\delta\theta}=0} 
& \left.\dfrac{\partial{}^{\mathcal{I}}\bm{p}_c}{\partial\bm{\delta{p}}} 
\right|_{\bm{\delta{p}}=0} \end{bmatrix}\\
&= \bm{K} \cdot 
\begin{bmatrix} 
\left.\dfrac{\partial}{\partial\bm{\delta\theta}}
\left( \bm{I}_3 + \begin{bmatrix} \bm{\delta\theta} \end{bmatrix}_{\times} \right) \cdot \bm{R}_{CM} \cdot {}^M\bm{p}_c
\right|_{\bm{\delta\theta}=0}
&
\left.\dfrac{\partial\bm{\delta{p}}}{\partial\bm{\delta{p}}}
\right|_{\bm{\delta{p}}=0}
\end{bmatrix}\\
&= \bm{K}\cdot\begin{bmatrix} 
\left.\dfrac{\partial}{\partial\bm{\delta\theta}} \begin{bmatrix} \bm{\delta\theta} \end{bmatrix}_{\times} \bm{R}_{CM} {}^M\bm{p}_c \right|_{\bm{\delta\theta}=0}
& \bm{I}_3 \end{bmatrix}\\
&= \bm{K} \cdot \begin{bmatrix} - \left. \dfrac{\partial}{\partial\bm{\delta\theta}} 
\begin{bmatrix} \bm{R}_{CM} \cdot {}^M\bm{p}_c \end{bmatrix}_{\times} \bm{\delta\theta} 
\right|_{\bm{\delta\theta}=0}
& \bm{I}_3\end{bmatrix}\\
&= \bm{K} \cdot \begin{bmatrix} - \begin{bmatrix} \bm{R}_{CM}{}^M\bm{p}_c \end{bmatrix}_{\times} & \bm{I}_3 \end{bmatrix}
\end{align}
\end{subequations}
Cette matrice Jacobienne permet de passer en coordonnées homogènes. Pour passer aux coordonnées 
3D, il faut diviser par la dernière composante :
\begin{align}
{}^{\mathcal{I}}\bm{p}_c = \begin{bmatrix} x \\ y \end{bmatrix} = \begin{bmatrix} 
\dfrac{\tilde{x}}{\tilde{z}}, \dfrac{\tilde{y}}{\tilde{z}} 
\end{bmatrix}^{\top}
\end{align}
La matrice Jacobienne correspondant à cette opération est:
\begin{align}
\bm{J}^{{}^{\mathcal{I}}\bm{p}_c}_{{}^{\mathcal{I}}\tilde{\bm{p}}_c} = \left. 
\dfrac{\partial{}^{\mathcal{I}}\bm{p}_c}{\partial{}^{\mathcal{I}}\tilde{\bm{p}}_c}\right|_{\tilde{\bm{p}}} =
\begin{bmatrix} 1/\tilde{z} & 0 & - \tilde{x}/\tilde{z}^2 \\ 0 & 1/\tilde{z} & 
-\tilde{y}/\tilde{z}^2 \end{bmatrix} =
\dfrac{1}{\tilde{z}} \begin{bmatrix} 1 & 0 & - \tilde{x}/\tilde{z} \\ 0 & 1 & 
- \tilde{y}/\tilde{z} \end{bmatrix}
\end{align}
La matrice Jacobienne de la projection totale vaut alors :
\begin{align}
\bm{J}^{{}^{\mathcal{I}}\bm{p}_c}_{T_{CM}}(\hat{T}_{CM}) =
\bm{J}^{{}^{\mathcal{I}}\bm{p}_c}_{{}^{\mathcal{I}}\tilde{\bm{p}}_c} \cdot
\bm{K} \cdot \begin{bmatrix} - \begin{bmatrix} \hat{\bm{R}}_{CM} {}^M\hat{\bm{p}}_c \end{bmatrix}_{\times}  & \bm{I}_3 \end{bmatrix}
\end{align}
En supposant un écart type $\sigma_{\text{vis}}$ sur chaque pixel, la matrice Fisher vaut :
\begin{align}
\boxed{\bm{\mathcal{I}}_{\mathcal{Z}}(\hat{T}_{CM}) = \dfrac{1}{\sigma_{\text{vis}}^2} \displaystyle\sum_{i=1}^4 
\left( \bm{J}^{{}^{\mathcal{I}}\bm{p}_{c_i}}_{T_{CM}}(\hat{T}_{CM}) \right)^{\top} 
\left( \bm{J}^{{}^{\mathcal{I}}\bm{p}_{c_i}}_{T_{CM}}(\hat{T}_{CM}) \right)}
\end{align}

\section{Homographie induite par un plan}

On considère un plan dont l'équation homogène est :
\begin{align}
{}^{C}\bm{\pi}^{\top} \cdot {}^{C}\bm{X} = 0
\end{align}
avec ${}^C\bm{\pi} = \begin{bmatrix} {}^C\bm{n}^{\top} & d_C \end{bmatrix}^{\top}$. Le point dans le 
plan focal :
\begin{align}
{}^F\bm{x} = \begin{bmatrix} \bm{I} & \bm{0} \end{bmatrix} {}^C\bm{X}
\end{align}
Pour que le point rétro-projeté appartienne au plan, il doit être de la forme :
\begin{align}
{}^C\bm{X} = \begin{bmatrix} {}^F\bm{x} \\ - \dfrac{{}^C\bm{n}^{\top} \cdot {}^F\bm{x}}{d_C} \end{bmatrix}
\end{align}
En projetant ceci dans une autre caméra :
\begin{align}
{}^{F'}\bm{x}^{'} = \begin{bmatrix} \bm{R}_{C'C} & {}^{C'}\bm{t}_C \end{bmatrix} {}^C\bm{X}
= \left( \bm{R}_{C'C} - \dfrac{{}^{C'}\bm{t}_C.{}^C\bm{n}^{\top}}{d} \right) {}^F\bm{x}
\end{align}
En ajoutant les matrices des caméras :
\begin{align}
\bm{x}^{'} = \bm{K}^{'} \left( \bm{R}_{C'C} - \dfrac{{}^{C'}\bm{t}_C.{}^C\bm{n}^{\top}}{d_C} \right) 
\bm{K}^{-1} \bm{x}
\end{align}
En considérant la transformation qui permet de passer de la caméra de référence à la caméra 
désirée :
\begin{align}
\boxed{\bm{x}^{'} = \bm{K}^{'} \bm{R}_{CC'}^{\top} \left( \bm{I}_3 +  
\dfrac{{}^C\bm{t}_{C'}.{}^C\bm{n}^{\top}}{d_C} \right) \bm{K}^{-1} \bm{x}}
\end{align}
Pour simplifier, on peut considérer un vecteur ${}^C\tilde{\bm{\pi}}$ normalisé :
\begin{align}
\boxed{{}^{I'}\bm{x} = \bm{K}_{I'C'} \left( \bm{R}_{C'C} -  
{}^{C'}\bm{t}_{C'C}.{}^C\tilde{\bm{n}}^{\top} \right) \bm{K}_{IC}^{-1}.{}^I\bm{x}}
\end{align}
Si on connaît le vecteur normal dans le référentiel monde :
\begin{align}
{}^{I'}\bm{x} &= \bm{K}_{I'C'} \left( \bm{R}_{C'C} -  
{}^{C'}\bm{t}_{C'C}.(\bm{R}_{CW}.{}^W\tilde{\bm{n}})^{\top} \right) \bm{K}_{IC}^{-1}.{}^I\bm{x}\\
&= \bm{K}_{I'C'} \left( \bm{R}_{C'C} -  
{}^{C'}\bm{t}_{C'C}.{}^W\tilde{\bm{n}}^{\top}.\bm{R}_{WC} \right) \bm{K}_{IC}^{-1}.{}^I\bm{x}\\
&= \bm{K}_{I'C'} \left( \bm{R}_{C'W} -  
{}^{C'}\bm{t}_{C'C}.{}^W\tilde{\bm{n}}^{\top} \right) \bm{R}_{WC}.\bm{K}_{IC}^{-1}.{}^I\bm{x}\\
\end{align}
Changement de référentiel (axes + orgine) :
\begin{align}
{}^{A'}\bm{t}_{A'C} = {}^{A'}\bm{t}_{A'A} + \underset{={}^{A'}\bm{t}_{AC}}{\underbrace{\bm{R}_{A'A}.{}^A\bm{t}_{AC}}}
\end{align}

\section{Modèle de distortion d'OpenCV}

Le modèle de distortion d'OpenCV est de la forme suivante :
\begin{align*}
\left\{\begin{array}{l}
x_d = \dfrac{1 + k_1r^2 + k_2r^4 + k_3r^6}{1 + k_4 r^2 + k_5r^4 + k_6r^6}x + 2\rho_1xy + \rho_2(r^2 + 2x^2)\\[1em]
y_d = \dfrac{1 + k_1r^2 + k_2r^4 + k_3r^6}{1 + k_4 r^2 + k_5r^4 + k_6r^6}y + \rho_1(r^2 + 2y^2) + 2\rho_2xy
\end{array}\right.
\end{align*}
Ainsi, le vecteur classique des paramètres de distortion utilisé est de la forme :
\begin{align*}
k_1, k_2, \rho_1, \rho_2, k_3, k_4, k_5, k_6, s_1, s_2, s_3, s_4
\end{align*}
Attention, le modèle de distortion Fisheye n'est pas le même, et n'est pas géré par le module Aruco :
\begin{align*}
\left\{\begin{array}{l}
x_d = \dfrac{\theta}{r}(1 + k_1\theta^2 + k_2\theta^4 + k_3\theta^6 + k_4\theta^8)x\\[1em]
y_d = \dfrac{\theta}{r}(1 + k_1\theta^2 + k_2\theta^4 + k_3\theta^6 + k_4\theta^8)y
\end{array}\right.
\end{align*}
Donc, non seulement ce ne sont pas les mêmes coefficients, mais les polynômes de distorsion ne portent
pas non plus sur les mêmes variables ! Mais on peut utiliser la forme à cinq coefficients en premier
spécifiquement pour les caméras fisheye, cela devrait suffire, mais il faut tester !

\end{document}
